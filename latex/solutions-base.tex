\usepackage[T1, OT1]{fontenc}
\DeclareTextSymbolDefault{\dh}{T1}
\usepackage[english]{babel}
\usepackage{lmodern}

%-------------------------------------------------------------------------------
% The following are required for most problems:
%-------------------------------------------------------------------------------
\usepackage{amsmath,amssymb}
\usepackage{pgf,tikz}
\usepackage{mathrsfs}
\usetikzlibrary{arrows}
\usetikzlibrary{shapes}
\usetikzlibrary{backgrounds}
\usetikzlibrary{patterns}
\usetikzlibrary{positioning}
\usepackage{pgfplots}
\usepackage{pgfplotstable}
\pgfplotsset{compat=1.15}
\usepackage{graphicx}
\usepackage{listings}
%\usepackage{subcaption}
\usepackage{algorithm}
\usepackage[makeroom]{cancel}
\usepackage[noend]{algpseudocode}
\usepackage{standalone}
\usepackage{ifthen}
\usepackage{tcolorbox}
\usepackage[autoplay,controls,loop,poster=last]{animate}

\providecolor{Accepted}{named}{LimeGreen}
\providecolor{WrongAnswer}{named}{Red}
\providecolor{TimeLimit}{named}{Plum}
\providecolor{RunError}{named}{Goldenrod}
\providecolor{Pending}{named}{SkyBlue}

\lstset{
	backgroundcolor=\color{white},
	tabsize=4,
	language=python,
	basicstyle=\footnotesize\ttfamily,
	breaklines=true,
	keywordstyle=\color[rgb]{0, 0, 1},
	commentstyle=\color[rgb]{0, 0.5, 0},
	stringstyle=\color{red}
}

%-------------------------------------------------------------------------------
% General layout settings:
%
% These are very similar to the setting from the BAPC 2015 presentation, except
% that the colours have been replaced by the TU Delft colours.
%-------------------------------------------------------------------------------
\usetheme{default}
\useinnertheme{rectangles}
\useoutertheme{infolines}
\setbeamertemplate{navigation symbols}{} % remove navigation symbols
\setbeamertemplate{frametitle}[default][center]
\setbeamersize{text margin left=1cm}
\setbeamersize{text margin right=1cm}

\definecolor{tudCyan}{RGB}{61,152,222} % TU Delft colour
\definecolor{uva}{RGB}{64,83,163} % Bapc 2017 UvA colour
\definecolor{beamer@blendedblue}{RGB}{64,83,163}
\definecolor{uclouvain}{RGB}{0,46,98} % BAPC 2018 Louvain
\definecolor{rucolour}{RGB}{0,46,98} % BAPC 2019 Nijmegen
\definecolor{titlegray}{RGB}{60,60,60} % BAPC 2019 Nijmegen

\setbeamercolor{footlinecolor}{fg=white,bg=rucolour}
\setbeamercolor{headlinecolor}{bg=titlegray}
\setbeamercolor{frametitle}{fg=rucolour}
\setbeamercolor{title}{fg=rucolour}

%\setbeamertemplate{footline}{%
	%\begin{beamercolorbox}[sep=1em,wd=\paperwidth,leftskip=0.5cm,rightskip=0.5cm]{footlinecolor}
		%\insertshorttitle\ --- \insertdate \hfill\hfill\insertframenumber
	%\end{beamercolorbox}%
%}
%\setbeamertemplate{headline}{%
	%\begin{beamercolorbox}[sep=2.4em,wd=\paperwidth,leftskip=0.5cm,rightskip=0.5cm]{headlinecolor}
		%\insertshorttitle\ --- \insertdate \hfill\hfill\insertframenumber
	%\end{beamercolorbox}%
%}

%-------------------------------------------------------------------------------
% BAPC logo in the top right corner:
%-------------------------------------------------------------------------------
%\addtobeamertemplate{headline}{
	%\begin{figure}[!ht]
		%\vspace{0.02\textwidth}
		%\hspace{0.05\textwidth}
		%\includegraphics[width=0.14\textwidth]{logo}
		%\hfill
		%\phantom{}
	%\end{figure}
	%\vspace{-0.10\textwidth}
%}

% A bit ugly solution, but it works...
\setbeamertemplate{frametitle}{%
  \nointerlineskip%
  \begin{beamercolorbox}[%
      wd=\paperwidth,%
      sep=0pt,%
      leftskip=4.2ex,%
      rightskip=2.2ex,%
    ]{frametitle}%
    \begin{minipage}{0.4\paperwidth}%
    \rule{0pt}{2.2ex}%
	\ifdefempty{\problemlabel}{%
		\insertframetitle%
	}{%
		\fullproblemtitle %
		\\[0.3em]%
		\tiny%
		Problem Author: \problemauthor%
	}%
    \nolinebreak%
    \end{minipage}%
    \begin{minipage}{0.5\paperwidth}%
      \activitychart%
    \end{minipage}%
    \rule[-2.2ex]{0pt}{2.2ex}%
  \end{beamercolorbox}%
}

\newcommand{\timelimit}{1.0s}
\newcommand{\problemlabel}{} % Empty to hide activity chart
\newcommand{\problemyamlname}{Problem name}
\newcommand{\problemauthor}{Problem author}
\newcommand{\fullproblemtitle}{\problemlabel: \problemyamlname}

% If solve_stats/activity/A.tex exists, define the \activitychart command
\IfFileExists{A.tex}{
	\newcommand{\activitychart}{
	  \ifdefempty{\problemlabel}{}{
		\includestandalone[width=\textwidth]{solve_stats/activity/\problemlabel}
		\vspace{-2.5em}
	  }
	}
}{
	\newcommand{\activitychart}{}
}

\newcommand{\printsolvestats}[3]{%
	\vfill
	Statistics: #1 submissions, #2 accepted, #3 unknown%
}

% Define \solvestats for the current problem if the file exists.
\IfFileExists{problem_stats.tex}{
	\newcommand{\solvestats}{\csname solvestats\problemlabel \endcsname}
	\providecommand{\solvestatsA}{\printsolvestats{42}{0}{17}}

}{
	\newcommand{\solvestats}{}
}
