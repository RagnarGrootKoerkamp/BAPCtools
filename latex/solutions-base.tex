\usepackage[T1, OT1]{fontenc}
\DeclareTextSymbolDefault{\dh}{T1}
\usepackage[english]{babel}
\usepackage{lmodern}

%-------------------------------------------------------------------------------
% The following are required for most problems:
%-------------------------------------------------------------------------------
\usepackage{amsmath,amssymb}
\usepackage{pgf,tikz}
\usepackage{mathrsfs}
\usetikzlibrary{arrows}
\usetikzlibrary{shapes}
\usetikzlibrary{backgrounds}
\usetikzlibrary{patterns}
\usetikzlibrary{positioning}
\usepackage{pgfplots}
\usepackage{pgfplotstable}
\pgfplotsset{compat=1.15}
\usepackage{graphicx}
\usepackage{listings}
%\usepackage{subcaption}
\usepackage{algorithm}
\usepackage[makeroom]{cancel}
\usepackage[noend]{algpseudocode}
\usepackage{standalone}
\usepackage{ifthen}
\usepackage{tcolorbox}
\usepackage[autoplay,controls,loop,poster=last]{animate}
\usepackage{multicol}
\usepackage{wrapfig}

\lstset{
    backgroundcolor=\color{white},
    tabsize=4,
    language=python,
    basicstyle=\footnotesize\ttfamily,
    breaklines=true,
    keywordstyle=\color[rgb]{0, 0, 1},
    commentstyle=\color[rgb]{0, 0.5, 0},
    stringstyle=\color{red}
}

\newcommand{\timelimit}{0.0s}
\newcommand{\problemlabel}{} % Empty to hide activity chart
\newcommand{\problemauthor}{Problem author}
% TODO: Clean these up
\newcommand{\problemyamlname}{Problem name}
\newcommand{\fullproblemtitle}{\problemlabel: \problemyamlname}
\newcommand{\problemtitle}{\problemlabel: \problemyamlname}

% If solve_stats/activity/A.pdf exists, define the \activitychart command
\IfFileExists{solve_stats/activity/A.pdf}{
    \newcommand{\activitychart}{
      \ifdefempty{\problemlabel}{}{%
        \includegraphics[width=\textwidth,height=0.1\textheight]{solve_stats/activity/\problemlabel}%
      }%
    }
}{
    \newcommand{\activitychart}{}
}

\newcommand{\printsolvestats}[3]{%
    \vfill%
    \onslide<+->%
    Statistics: #1 submissions, #2 accepted\ifthenelse{\equal{#3}0}{}{, #3 unknown}%
}

% Define \solvestats for the current problem if the file exists.
\IfFileExists{problem_stats.tex}{
    \newcommand{\solvestats}{\csname solvestats\problemlabel \endcsname}
    \providecommand{\solvestatsA}{\printsolvestats{42}{0}{17}}

}{
    % If the file does not exist, use a placeholder.
    \newcommand{\solvestats}{\printsolvestats{\ldots}{\ldots}{\ldots}}
}

\usetheme[numbering=none,block=fill]{metropolis}

\setbeamertemplate{frametitle}{%
  \nointerlineskip%
  \begin{beamercolorbox}[%
      wd=\paperwidth,%
      sep=0.4ex,%
      leftskip=4.2ex,% 2.2ex
      rightskip=2.2ex,%
    ]{frametitle}%
    \begin{minipage}{0\paperwidth}%
    \rule{0ex}{5.5ex}%
    \end{minipage}
    \begin{minipage}{0.4\paperwidth}%
    \ifdefempty{\problemlabel}{%
        \insertframetitle\strut%
    }{%
        \problemtitle%
        \\[0.3em]%
        \tiny%
        Problem Author: \problemauthor%
        \strut%
    }%
    \end{minipage}%
    \hfill%
    \begin{minipage}{0.5\paperwidth}%
      \activitychart%
      %\vspace{-2ex}%
    \end{minipage}%
    %\rule[-2.2ex]{0pt}{2.2ex}%
  \end{beamercolorbox}%
}
