\usepackage[T1, OT1]{fontenc}
\DeclareTextSymbolDefault{\dh}{T1}
\usepackage[english]{babel}
\usepackage{lmodern}

%-------------------------------------------------------------------------------
% The following are required for most problems:
%-------------------------------------------------------------------------------
\usepackage{amsmath,amssymb}
\usepackage{pgf,tikz}
\usepackage{mathrsfs}
\usetikzlibrary{arrows}
\usetikzlibrary{shapes}
\usetikzlibrary{backgrounds}
\usetikzlibrary{patterns}
\usetikzlibrary{positioning}
\usepackage{pgfplots}
\usepackage{pgfplotstable}
\pgfplotsset{compat=1.15}
\usepackage{graphicx}
\usepackage{listings}
%\usepackage{subcaption}
\usepackage{algorithm}
\usepackage[makeroom]{cancel}
\usepackage[noend]{algpseudocode}
\usepackage{standalone}
\usepackage{ifthen}
\usepackage{tcolorbox}
\usepackage{upquote}          % For ' in samples
\usepackage[autoplay,controls,loop,poster=last]{animate}

\lstset{
  backgroundcolor=\color{white},
  tabsize=4,
  language=python,
  basicstyle=\footnotesize\ttfamily,
  breaklines=true,
  keywordstyle=\color[rgb]{0, 0, 1},
  commentstyle=\color[rgb]{0, 0.5, 0},
  stringstyle=\color{red}
}

\newcommand{\timelimit}{0}
\newcommand{\problemlabel}{} % Empty to hide activity chart
\newcommand{\problemauthor}{Problem author}
\newcommand{\problembackground}{}
\newcommand{\problemforeground}{}
\newcommand{\problemborder}{}
% TODO: Clean these up
\newcommand{\problemyamlname}{Problem name}
\newcommand{\fullproblemtitle}{\problemlabel: \problemyamlname}
\newcommand{\problemtitle}{\problemyamlname}

\newcommand{\constant}[1]{%
    \ifcsname constants_#1\endcsname%
        \csname constants_#1\endcsname%
    \else%
        \PackageError{constants}{constant{#1} is not defined}{}%
    \fi%
}

\usetheme[numbering=none,block=fill]{metropolis}

\newcommand{\illustration}[3]{
  \begin{column}[T]{#1\textwidth}
    \includegraphics[width=\textwidth]{#2}
    \ifstrempty{#3}{
      \vspace{-5pt}
    }{
      \begin{flushright}
        \vspace{-5pt}
        \tiny #3
      \end{flushright}
    }
  \end{column}
}

\setbeamertemplate{frametitle}{%
    \nointerlineskip%
    \vspace{1em}%
    \begin{minipage}{0.06\paperwidth}%
        \begin{tikzpicture}
            \definecolor{problembackground}{HTML}{\problembackground}
            \definecolor{problemforeground}{HTML}{\problemforeground}
            \definecolor{problemborder}{HTML}{\problemborder}
            % Hack: the square for problem label "M" would be too wide so use `\huge` instead of `\Huge`
            \node[rectangle,rounded corners,thick,minimum size=2em,draw=problemborder,fill=problembackground,text=problemforeground] (0, 0) {\if M\problemlabel\huge\else\Huge\fi\problemlabel};
        \end{tikzpicture}
    \end{minipage}%
    \begin{minipage}{0.8\paperwidth}%
        \color{black}%
        \ifdefempty{\problemlabel}{%
            \insertframetitle\strut%
        }{%
            \problemtitle%
            \\[0.3em]%
            \tiny%
            Time limit: \timelimit{}s%
            \quad\quad%
            Problem Author: \problemauthor%
            \strut%
        }%
    \end{minipage}%
    \hfill%
}
